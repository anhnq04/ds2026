\documentclass[12pt,a4paper]{article}
\usepackage[utf8]{inputenc}
\usepackage[T1]{fontenc}
\usepackage{graphicx}
\usepackage{listings}
\usepackage{xcolor}
\usepackage{hyperref}
\usepackage{geometry}
\geometry{margin=2.5cm}

% Python code configuration
\lstset{
    language=Python,
    basicstyle=\ttfamily\small,
    keywordstyle=\color{blue},
    commentstyle=\color{green!60!black},
    stringstyle=\color{red},
    numbers=left,
    numberstyle=\tiny\color{gray},
    stepnumber=1,
    numbersep=8pt,
    backgroundcolor=\color{gray!10},
    frame=single,
    breaklines=true,
    captionpos=b
}

\title{\textbf{PRACTICAL WORK 3} \\ 
       \Large MPI File Transfer}

\begin{document}

\maketitle
\newpage

\tableofcontents
\newpage

\section{Objective}
Upgrade the TCP file transfer system from Lab 1 to use MPI (Message Passing Interface):
\begin{itemize}
    \item Copy TCP file transfer system to MPI directory
    \item Upgrade to MPI-based file transfer
    \item Use mpi4py implementation
    \item Enable parallel processing and inter-process communication
\end{itemize}

\section{Why MPI?}

\subsection{What is MPI?}
Message Passing Interface (MPI) is a standardized and portable message-passing system designed for parallel computing. It enables multiple processes to communicate and coordinate their work.

\subsection{Why Choose mpi4py?}
\begin{itemize}
    \item \textbf{Python Integration}: Native Python bindings for MPI
    \item \textbf{Standard Compliance}: Implements MPI-2 and MPI-3 standards
    \item \textbf{High Performance}: Near C-level performance for large data transfers
    \item \textbf{Easy to Use}: Pythonic API with pickle support for Python objects
    \item \textbf{Widely Used}: Industry standard for HPC (High Performance Computing)
    \item \textbf{Scalable}: Can run on single machine or across clusters
\end{itemize}

\subsection{MPI vs RPC vs TCP}
\begin{center}
\begin{tabular}{|l|l|l|l|}
\hline
\textbf{Aspect} & \textbf{TCP} & \textbf{RPC} & \textbf{MPI} \\
\hline
Paradigm & Client-Server & Client-Server & Peer-to-Peer \\
Processes & 2 (1 server, 1 client) & 2 (1 server, 1 client) & N (all equal) \\
Communication & Socket & HTTP/XML & Direct message passing \\
Performance & Good & Moderate & Excellent \\
Scalability & Limited & Limited & High \\
Use Case & Network apps & Web services & HPC/Scientific \\
\hline
\end{tabular}
\end{center}

\section{MPI Service Design}

\subsection{Process Roles}
In our implementation:
\begin{itemize}
    \item \textbf{Rank 0 (Master)}: Reads file from disk and sends to rank 1
    \item \textbf{Rank 1 (Worker)}: Receives file and saves to disk
\end{itemize}

Note: Unlike client-server, all MPI processes are peers. We assign roles based on rank.

\subsection{Communication Pattern}
\begin{enumerate}
    \item Launch 2 MPI processes using \texttt{mpiexec}
    \item Rank 0 reads filename from command line
    \item Rank 0 reads file content into memory
    \item Rank 0 sends dictionary \{filename, data\} to rank 1 using \texttt{comm.send()}
    \item Rank 1 receives message using \texttt{comm.recv()}
    \item Rank 1 saves file with prefix "received\_"
    \item Both processes print status messages
\end{enumerate}

\subsection{Technical Specifications}
\begin{itemize}
    \item \textbf{MPI Implementation}: MPICH or OpenMPI
    \item \textbf{Python Library}: mpi4py
    \item \textbf{Number of Processes}: 2
    \item \textbf{Communication}: Point-to-point (send/recv)
    \item \textbf{Data Serialization}: Pickle (automatic)
    \item \textbf{Transport}: Shared memory or network (MPI decides)
\end{itemize}

\section{System Organization}

\subsection{Directory Structure}
\begin{verbatim}
Lab3/
├── transfer.py                # MPI file transfer program
├── test.txt                   # Test file
├── README.md                  # Installation guide
└── 03.mpi.file.transfer.tex   # LaTeX report
\end{verbatim}

\subsection{System Architecture}
\begin{itemize}
    \item \textbf{Single Program, Multiple Data (SPMD)}: Same code runs on all processes
    \item \textbf{Rank-based Logic}: Different behavior based on process rank
    \item \textbf{MPI Communicator}: \texttt{MPI.COMM\_WORLD} manages all processes
    \item \textbf{Automatic Serialization}: mpi4py handles Python object serialization
\end{itemize}

\section{Implementation}

\subsection{MPI Transfer Code}
File \texttt{transfer.py} implements the MPI file transfer:

\lstinputlisting[caption=transfer.py]{transfer.py}

\subsection{Code Explanation}

\subsubsection{Initialization (Lines 1-6)}
\begin{itemize}
    \item \textbf{Line 1}: Import mpi4py MPI module
    \item \textbf{Line 4}: Get MPI communicator (all processes)
    \item \textbf{Line 5}: Get process rank (0, 1, 2, ...)
    \item \textbf{Line 6}: Get total number of processes
\end{itemize}

\subsubsection{Master Process - Rank 0 (Lines 8-23)}
\begin{itemize}
    \item \textbf{Lines 9-12}: Check command line arguments
    \item \textbf{Lines 15-17}: Read file content
    \item \textbf{Line 18}: Print sending status
    \item \textbf{Line 19}: Send dictionary to rank 1 using MPI
    \item \textbf{Lines 21-23}: Handle file not found error
\end{itemize}

\subsubsection{Worker Process - Rank 1 (Lines 24-35)}
\begin{itemize}
    \item \textbf{Line 26}: Receive message from rank 0
    \item \textbf{Lines 27-32}: Extract data and save file
    \item \textbf{Line 30}: Save with "received\_" prefix
    \item \textbf{Line 32}: Print success message
\end{itemize}

\subsection{Key MPI Concepts Used}
\begin{itemize}
    \item \textbf{Communicator}: \texttt{MPI.COMM\_WORLD} - group of all processes
    \item \textbf{Rank}: Unique ID for each process (0 to size-1)
    \item \textbf{Point-to-point}: \texttt{send()} and \texttt{recv()} for direct communication
    \item \textbf{Pickle Protocol}: Automatic serialization of Python objects
\end{itemize}

\section{User Guide}

\subsection{System Requirements}
\begin{itemize}
    \item Python 3.x
    \item MPI implementation (MPICH or OpenMPI)
    \item mpi4py library
\end{itemize}

\subsection{Installation}

\textbf{Ubuntu/Debian}:
\begin{verbatim}
sudo apt-get update
sudo apt-get install mpich
pip install mpi4py
\end{verbatim}

\textbf{Or using OpenMPI}:
\begin{verbatim}
sudo apt-get install openmpi-bin openmpi-common libopenmpi-dev
pip install mpi4py
\end{verbatim}

\subsection{How to Run}

\textbf{Basic Usage}:
\begin{verbatim}
cd ds2026/Lab3
mpiexec -n 2 python transfer.py test.txt
\end{verbatim}

\textbf{Parameters}:
\begin{itemize}
    \item \texttt{-n 2}: Launch 2 processes
    \item \texttt{test.txt}: File to transfer
\end{itemize}

\subsection{Expected Results}
\begin{verbatim}
[Rank 0] Sending file: test.txt (123 bytes)
[Rank 0] File sent successfully!
[Rank 1] File received and saved as: received_test.txt (123 bytes)
\end{verbatim}

\section{Experimental Results}

\subsection{Test Case 1: Text File Transfer}
\begin{itemize}
    \item \textbf{Input}: test.txt (text file)
    \item \textbf{Command}: \texttt{mpiexec -n 2 python transfer.py test.txt}
    \item \textbf{Result}: Success
    \item \textbf{Observation}: File transferred via MPI message passing
\end{itemize}

\subsection{Test Case 2: Binary File Transfer}
\begin{itemize}
    \item \textbf{Input}: image.png (binary file)
    \item \textbf{Command}: \texttt{mpiexec -n 2 python transfer.py image.png}
    \item \textbf{Result}: Success
    \item \textbf{Observation}: Binary data preserved correctly
\end{itemize}

\subsection{Test Case 3: Large File Transfer}
\begin{itemize}
    \item \textbf{Input}: large.zip (50MB)
    \item \textbf{Result}: Success
    \item \textbf{Observation}: MPI handles large messages efficiently
\end{itemize}

\subsection{Test Case 4: Non-existent File}
\begin{itemize}
    \item \textbf{Input}: nonexistent.txt
    \item \textbf{Result}: Error message displayed
    \item \textbf{Observation}: Proper error handling
\end{itemize}

\section{Who Does What}

\subsection{Rank 0 (Master) Responsibilities}
\begin{itemize}
    \item Parse command line arguments
    \item Read file from filesystem
    \item Package data into dictionary
    \item Send message to rank 1
    \item Handle file not found errors
    \item Print status messages
\end{itemize}

\subsection{Rank 1 (Worker) Responsibilities}
\begin{itemize}
    \item Wait for message from rank 0
    \item Receive and unpack data
    \item Save file to disk with new name
    \item Print confirmation message
\end{itemize}

\subsection{MPI Runtime Responsibilities}
\begin{itemize}
    \item Launch multiple processes
    \item Establish communication channels
    \item Serialize/deserialize Python objects
    \item Route messages between processes
    \item Synchronize process execution
    \item Clean up resources on exit
\end{itemize}

\section{Conclusion}

\subsection{Advantages of MPI Approach}
\begin{itemize}
    \item \textbf{High Performance}: Optimized for parallel computing
    \item \textbf{Scalability}: Can scale to thousands of processes
    \item \textbf{Flexibility}: Supports various communication patterns
    \item \textbf{Standard}: Industry standard for HPC
    \item \textbf{Automatic Optimization}: MPI chooses best transport (shared memory vs network)
    \item \textbf{Peer-to-Peer}: No client-server bottleneck
\end{itemize}

\subsection{Disadvantages of MPI Approach}
\begin{itemize}
    \item \textbf{Complex Setup}: Requires MPI installation
    \item \textbf{Learning Curve}: MPI concepts are more complex
    \item \textbf{Overkill for Simple Tasks}: Too powerful for basic file transfer
    \item \textbf{Process Management}: Must manage multiple processes
    \item \textbf{Debugging}: Harder to debug parallel programs
\end{itemize}

\subsection{Lessons Learned}
\begin{itemize}
    \item Understanding MPI fundamentals (rank, communicator, send/recv)
    \item Learning parallel programming concepts
    \item Comparing different communication paradigms
    \item Using mpi4py for Python parallel computing
    \item Appreciating trade-offs between simplicity and performance
\end{itemize}

\subsection{Comparison Summary}

\begin{center}
\begin{tabular}{|l|c|c|c|}
\hline
\textbf{Feature} & \textbf{TCP} & \textbf{RPC} & \textbf{MPI} \\
\hline
Code Complexity & Medium & Low & Medium \\
Setup Complexity & Low & Low & High \\
Performance & Good & Moderate & Excellent \\
Scalability & 2 nodes & 2 nodes & N nodes \\
Best For & Networks & Web services & HPC \\
\hline
\end{tabular}
\end{center}

\subsection{When to Use MPI}
\begin{itemize}
    \item \textbf{Use MPI when}:
        \begin{itemize}
            \item Building HPC applications
            \item Need to scale to many processes
            \item Require maximum performance
            \item Working in scientific computing
            \item Have complex parallel algorithms
        \end{itemize}
    \item \textbf{Don't use MPI when}:
        \begin{itemize}
            \item Building simple client-server apps
            \item Users don't have MPI installed
            \item Only need 2 processes
            \item Web-based applications
        \end{itemize}
\end{itemize}

\subsection{Future Enhancements}
\begin{itemize}
    \item Implement collective operations (broadcast, scatter, gather)
    \item Add multiple workers for parallel processing
    \item Implement file chunking for distributed storage
    \item Add non-blocking communication for better performance
    \item Implement fault tolerance and error recovery
    \item Use MPI-IO for parallel file I/O
    \item Add performance benchmarking
\end{itemize}

\end{document}
