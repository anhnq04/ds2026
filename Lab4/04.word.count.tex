\documentclass[12pt,a4paper]{article}
\usepackage[utf8]{inputenc}
\usepackage[T1]{fontenc}
\usepackage{graphicx}
\usepackage{listings}
\usepackage{xcolor}
\usepackage{hyperref}
\usepackage{geometry}
\geometry{margin=2.5cm}

% Python code configuration
\lstset{
    language=Python,
    basicstyle=\ttfamily\small,
    keywordstyle=\color{blue},
    commentstyle=\color{green!60!black},
    stringstyle=\color{red},
    numbers=left,
    numberstyle=\tiny\color{gray},
    stepnumber=1,
    numbersep=8pt,
    backgroundcolor=\color{gray!10},
    frame=single,
    breaklines=true,
    captionpos=b
}

\title{\textbf{PRACTICAL WORK 4} \\ 
       \Large Word Count with MapReduce}

\begin{document}

\maketitle
\newpage

\tableofcontents
\newpage

\section{Objective}
Implement the classic Word Count example using MapReduce paradigm:
\begin{itemize}
    \item Create a custom MapReduce framework in Python
    \item Implement Mapper and Reducer for word counting
    \item Use parallel processing with multiprocessing
    \item Count word occurrences in large text files
\end{itemize}

\section{Why Python MapReduce?}

\subsection{Framework Choice}
I chose to implement a custom MapReduce framework in Python using \texttt{multiprocessing} because:

\begin{itemize}
    \item \textbf{Simplicity}: Python is concise and easy to understand
    \item \textbf{No External Dependencies}: Uses only Python standard library
    \item \textbf{Educational Value}: Building from scratch helps understand MapReduce internals
    \item \textbf{Flexibility}: Full control over implementation details
    \item \textbf{Parallel Processing}: \texttt{multiprocessing.Pool} provides true parallelism
    \item \textbf{Lightweight}: No need for Hadoop/Spark setup
\end{itemize}

\subsection{Alternatives Considered}
\begin{itemize}
    \item \textbf{Apache Hadoop}: Too heavy for simple tasks, requires Java
    \item \textbf{Apache Spark}: Better than Hadoop but still requires setup
    \item \textbf{mrjob}: Python library but adds dependency
    \item \textbf{Custom C++}: Higher performance but more complex
\end{itemize}

\section{MapReduce Architecture}

\subsection{Overall Workflow}

\begin{enumerate}
    \item \textbf{Input}: Read text file line by line
    \item \textbf{Map Phase}: Process each line in parallel
        \begin{itemize}
            \item Extract words using regex
            \item Emit (word, 1) pairs
        \end{itemize}
    \item \textbf{Shuffle \& Sort}: Group all values by key
        \begin{itemize}
            \item Collect all counts for each word
            \item Create (word, [1, 1, 1, ...]) pairs
        \end{itemize}
    \item \textbf{Reduce Phase}: Sum counts in parallel
        \begin{itemize}
            \item Sum all values for each key
            \item Emit (word, total\_count)
        \end{itemize}
    \item \textbf{Output}: Display sorted results
\end{enumerate}

\subsection{Parallelization Strategy}

\begin{itemize}
    \item \textbf{Map Parallelism}: Each line processed by different worker
    \item \textbf{Reduce Parallelism}: Each word's counts summed by different worker
    \item \textbf{Worker Pool}: Fixed number of workers (default: 4)
    \item \textbf{Load Balancing}: Automatic by \texttt{multiprocessing.Pool}
\end{itemize}

\section{Implementation}

\subsection{Complete Code}

\lstinputlisting[caption=wordcount.py]{wordcount.py}

\subsection{Mapper Function}

\begin{lstlisting}[caption=Mapper Implementation]
def mapper(line):
    """Map function: emit (word, 1) for each word in line"""
    words = re.findall(r'\b\w+\b', line.lower())
    return [(word, 1) for word in words]
\end{lstlisting}

\textbf{Explanation}:
\begin{itemize}
    \item \textbf{Input}: One line of text
    \item \textbf{Process}: 
        \begin{itemize}
            \item Convert to lowercase for case-insensitive counting
            \item Extract words using regex \texttt{\textbackslash b\textbackslash w+\textbackslash b}
            \item Emit (word, 1) for each word found
        \end{itemize}
    \item \textbf{Output}: List of (word, 1) tuples
\end{itemize}

\subsection{Reducer Function}

\begin{lstlisting}[caption=Reducer Implementation]
def reducer(word_counts):
    """Reduce function: sum all counts for a word"""
    word, counts = word_counts
    return (word, sum(counts))
\end{lstlisting}

\textbf{Explanation}:
\begin{itemize}
    \item \textbf{Input}: (word, [count1, count2, ...])
    \item \textbf{Process}: Sum all counts
    \item \textbf{Output}: (word, total\_count)
\end{itemize}

\subsection{Shuffle \& Sort Phase}

\begin{lstlisting}[caption=Shuffle Implementation]
shuffled = defaultdict(list)
for word, count in flat_mapped:
    shuffled[word].append(count)
\end{lstlisting}

\textbf{Explanation}:
\begin{itemize}
    \item Groups all (word, 1) pairs by word
    \item Creates (word, [1, 1, 1, ...]) structure
    \item Uses \texttt{defaultdict} for automatic list creation
\end{itemize}

\section{User Guide}

\subsection{System Requirements}
\begin{itemize}
    \item Python 3.x
    \item multiprocessing module (built-in)
    \item collections module (built-in)
    \item re module (built-in)
\end{itemize}

\subsection{How to Run}

\textbf{Basic Usage}:
\begin{verbatim}
cd ds2026/Lab4
python wordcount.py input.txt
\end{verbatim}

\textbf{Expected Output}:
\begin{verbatim}
Word Count Results:
----------------------------------------
swiss: 5
witch: 4
watch: 4
three: 3
...
\end{verbatim}

\section{Experimental Results}

\subsection{Test Case: Tongue Twister}

\textbf{Input} (input.txt):
\begin{verbatim}
three witches watch three swatch watches
which witch watches which swatch watch
three swiss switch bitches which wished to be 
switched swiss witch bitches watch three swiss 
swatch watch switches which swiss switch bitch
which wishes to be a switched witch bitch wishes 
to watch which swiss swatch watch switch
\end{verbatim}

\textbf{Results}:
\begin{verbatim}
swiss: 5
witch: 4
watch: 4
swatch: 4
which: 4
three: 3
bitches: 2
switched: 2
...
\end{verbatim}

\section{Who Does What}

\subsection{Main Process Responsibilities}
\begin{itemize}
    \item Read input file
    \item Create worker pool
    \item Distribute work to mappers
    \item Collect and shuffle results
    \item Distribute work to reducers
    \item Sort and display final results
\end{itemize}

\subsection{Mapper Workers}
\begin{itemize}
    \item Receive one line of text
    \item Extract words using regex
    \item Emit (word, 1) pairs
    \item Return results to main process
\end{itemize}

\subsection{Reducer Workers}
\begin{itemize}
    \item Receive (word, [counts]) pair
    \item Sum all counts
    \item Return (word, total) to main process
\end{itemize}

\subsection{Python Multiprocessing Framework}
\begin{itemize}
    \item Manage worker processes
    \item Handle inter-process communication
    \item Load balance work distribution
    \item Synchronize results collection
\end{itemize}

\section{Conclusion}

\subsection{Advantages}
\begin{itemize}
    \item \textbf{Simple Implementation}: ~60 lines of Python code
    \item \textbf{True Parallelism}: Uses multiple CPU cores
    \item \textbf{No Dependencies}: Only standard library
    \item \textbf{Scalable}: Can adjust number of workers
    \item \textbf{Educational}: Clear demonstration of MapReduce concepts
\end{itemize}

\subsection{Limitations}
\begin{itemize}
    \item \textbf{Single Machine}: Cannot distribute across cluster
    \item \textbf{Memory Bound}: All data must fit in memory
    \item \textbf{No Fault Tolerance}: Process crash = job failure
    \item \textbf{Limited Scalability}: Bounded by CPU cores
\end{itemize}

\subsection{Lessons Learned}
\begin{itemize}
    \item Understanding MapReduce paradigm deeply
    \item Implementing parallel processing in Python
    \item Trade-offs between simplicity and scalability
    \item Importance of shuffle/sort phase
    \item How to decompose problems into map/reduce operations
\end{itemize}

\subsection{Performance Analysis}

For a file with N lines and W workers:
\begin{itemize}
    \item \textbf{Map Phase}: O(N/W) - parallel speedup
    \item \textbf{Shuffle Phase}: O(N) - sequential bottleneck
    \item \textbf{Reduce Phase}: O(K/W) where K = unique words
    \item \textbf{Overall}: O(N + K/W) - good for large N
\end{itemize}

\subsection{Future Enhancements}
\begin{itemize}
    \item Add combiner function to reduce shuffle overhead
    \item Implement distributed version using network sockets
    \item Add progress reporting
    \item Support for multiple input files
    \item Implement partitioner for better load balancing
    \item Add fault tolerance with task retry
    \item Support streaming input for very large files
\end{itemize}

\end{document}
