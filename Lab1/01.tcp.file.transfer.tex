\documentclass[12pt,a4paper]{article}
\usepackage[utf8]{inputenc}
\usepackage[T1]{fontenc}
\usepackage{graphicx}
\usepackage{listings}
\usepackage{xcolor}
\usepackage{hyperref}
\usepackage{geometry}
\geometry{margin=2.5cm}

% Python code configuration
\lstset{
    language=Python,
    basicstyle=\ttfamily\small,
    keywordstyle=\color{blue},
    commentstyle=\color{green!60!black},
    stringstyle=\color{red},
    numbers=left,
    numberstyle=\tiny\color{gray},
    stepnumber=1,
    numbersep=8pt,
    backgroundcolor=\color{gray!10},
    frame=single,
    breaklines=true,
    captionpos=b
}

\title{\textbf{PRACTICAL WORK 1} \\ 
       \Large TCP File Transfer}

\begin{document}

\maketitle
\newpage

\tableofcontents
\newpage

\section{Objective}
Build a 1-to-1 file transfer system over TCP/IP using socket in Python, including:
\begin{itemize}
    \item One server to receive files
    \item One client to send files
    \item Using socket for data transmission
\end{itemize}

\section{Protocol Design}

\subsection{Workflow Diagram}
The file transfer protocol is designed as follows:

\begin{enumerate}
    \item Client establishes TCP connection to Server
    \item Client sends filename (text format, UTF-8 encoding)
    \item Client sends file content in chunks (4096 bytes)
    \item Client sends end-of-file signal "EOF"
    \item Server saves the file and closes connection
\end{enumerate}

\subsection{Technical Specifications}
\begin{itemize}
    \item \textbf{Protocol}: TCP/IP
    \item \textbf{Port}: 5000
    \item \textbf{Host}: 0.0.0.0 (Server), 127.0.0.1 (Client)
    \item \textbf{Buffer size}: 4096 bytes
    \item \textbf{Encoding}: UTF-8 (filename), Binary (file content)
\end{itemize}

\section{System Organization}

\subsection{Directory Structure}
\begin{verbatim}
Lab1/
├── server.py                    # File receiving server
├── client.py                    # File sending client
├── test.txt                     # Test file
└── 01.tcp.file.transfer.tex     # LaTeX report
\end{verbatim}

\subsection{System Architecture}
The system uses a simple Client-Server model:
\begin{itemize}
    \item \textbf{Server}: Listens for connections, receives files and saves to disk
    \item \textbf{Client}: Connects to server, reads file from disk and sends over network
\end{itemize}

\section{Implementation}

\subsection{Server Code}
File \texttt{server.py} implements the file receiving functionality:

\lstinputlisting[caption=server.py]{server.py}

\subsubsection{Server Code Explanation}
\begin{itemize}
    \item \textbf{Lines 1-2}: Import socket and os libraries
    \item \textbf{Lines 4-6}: Configure HOST, PORT and BUFFER size
    \item \textbf{Lines 8-10}: Create socket, bind address and listen
    \item \textbf{Lines 12-13}: Accept connection from client
    \item \textbf{Lines 15-17}: Receive filename from client
    \item \textbf{Lines 19-24}: Receive file content in chunks and save to disk
    \item \textbf{Lines 26-28}: Display success message and close connection
\end{itemize}

\subsection{Client Code}
File \texttt{client.py} implements the file sending functionality:

\lstinputlisting[caption=client.py]{client.py}

\subsubsection{Client Code Explanation}
\begin{itemize}
    \item \textbf{Lines 1-2}: Import socket and os libraries
    \item \textbf{Lines 4-6}: Configure HOST, PORT and BUFFER size
    \item \textbf{Line 8}: Get filename input from user
    \item \textbf{Lines 10-12}: Check if file exists
    \item \textbf{Lines 14-15}: Create socket and connect to server
    \item \textbf{Lines 17-18}: Send filename
    \item \textbf{Lines 20-25}: Read and send file content in chunks
    \item \textbf{Lines 27-29}: Send EOF signal and close connection
\end{itemize}

\section{User Guide}

\subsection{System Requirements}
\begin{itemize}
    \item Python 3.x
    \item socket module (built-in)
    \item os module (built-in)
\end{itemize}

\subsection{How to Run}

\textbf{Step 1}: Open first terminal, run Server:
\begin{verbatim}
cd Lab1
python server.py
\end{verbatim}

\textbf{Step 2}: Open second terminal, run Client:
\begin{verbatim}
cd Lab1
python client.py
\end{verbatim}

\textbf{Step 3}: Enter the filename to send (e.g., \texttt{test.txt})

\subsection{Expected Results}
\begin{itemize}
    \item Server displays: "Receiving: test.txt" and "File test.txt received successfully!"
    \item Client displays: "File test.txt sent successfully!"
    \item File is created in the Server directory
\end{itemize}

\section{Experimental Results}

\subsection{Test Case 1: Sending Text File}
\begin{itemize}
    \item \textbf{Input}: test.txt (regular text file)
    \item \textbf{Result}: Success
    \item \textbf{Observation}: File transferred completely, content is accurate
\end{itemize}

\subsection{Test Case 2: Non-existent File}
\begin{itemize}
    \item \textbf{Input}: nonexistent.txt
    \item \textbf{Result}: Client shows error "File not found!"
    \item \textbf{Observation}: Error handling works correctly
\end{itemize}

\subsection{Test Case 3: Binary File Transfer}
\begin{itemize}
    \item \textbf{Input}: image.png (binary file)
    \item \textbf{Result}: Success
    \item \textbf{Observation}: Binary files are transferred correctly without corruption
\end{itemize}

\section{Conclusion}

\subsection{Advantages}
\begin{itemize}
    \item Concise and easy-to-understand code (~30 lines per file)
    \item Uses TCP to ensure reliable data transmission
    \item Handles binary files correctly
    \item Includes basic error checking
    \item Simple implementation suitable for learning purposes
\end{itemize}

\subsection{Limitations and Future Improvements}
\begin{itemize}
    \item Only supports one client at a time
    \item No progress bar for large file transfers
    \item No data encryption
    \item No client authentication
    \item No resume capability for interrupted transfers
    \item No file integrity verification (checksum)
\end{itemize}

\subsection{Lessons Learned}
\begin{itemize}
    \item Understanding TCP socket operations
    \item Learning file transfer over network
    \item Grasping basic client-server architecture
    \item Handling binary data in Python
    \item Importance of proper error handling
\end{itemize}

\subsection{Future Enhancements}
To make this system production-ready, the following features could be added:
\begin{itemize}
    \item Multi-threaded server to handle multiple clients
    \item SSL/TLS encryption for secure transmission
    \item Authentication mechanism
    \item Progress bar and transfer speed indicator
    \item File compression before transfer
    \item MD5/SHA256 checksum verification
    \item Resume interrupted transfers
\end{itemize}

\end{document}
