\documentclass[12pt,a4paper]{article}
\usepackage[utf8]{inputenc}
\usepackage[T1]{fontenc}
\usepackage{graphicx}
\usepackage{listings}
\usepackage{xcolor}
\usepackage{hyperref}
\usepackage{geometry}
\geometry{margin=2.5cm}

% Python code configuration
\lstset{
    language=Python,
    basicstyle=\ttfamily\small,
    keywordstyle=\color{blue},
    commentstyle=\color{green!60!black},
    stringstyle=\color{red},
    numbers=left,
    numberstyle=\tiny\color{gray},
    stepnumber=1,
    numbersep=8pt,
    backgroundcolor=\color{gray!10},
    frame=single,
    breaklines=true,
    captionpos=b
}

\title{\textbf{PRACTICAL WORK 2} \\ 
       \Large RPC File Transfer}
\begin{document}

\maketitle
\newpage

\tableofcontents
\newpage

\section{Objective}
Upgrade the TCP file transfer system from Lab 1 to use RPC (Remote Procedure Call):
\begin{itemize}
    \item Copy TCP file transfer system to RPC directory
    \item Upgrade to RPC-based file transfer
    \item Use XML-RPC service (Python built-in)
    \item Maintain file transfer functionality with higher abstraction
\end{itemize}

\section{RPC Service Design}

\subsection{What is RPC?}
Remote Procedure Call (RPC) is a protocol that allows a program to execute procedures on a remote server as if they were local function calls. It abstracts the network communication details from the developer.

\subsection{Why XML-RPC?}
\begin{itemize}
    \item \textbf{Built-in}: Available in Python standard library
    \item \textbf{Simple}: Easy to implement and understand
    \item \textbf{Cross-platform}: Works across different systems
    \item \textbf{Language-agnostic}: Can communicate with other languages
\end{itemize}

\subsection{RPC Service Methods}
The server exposes one main RPC method:

\begin{verbatim}
upload_file(filename: str, data: str) -> str
\end{verbatim}

\begin{itemize}
    \item \textbf{filename}: Name of the file to save
    \item \textbf{data}: Base64-encoded file content
    \item \textbf{Returns}: Success or error message
\end{itemize}

\subsection{Protocol Workflow}
\begin{enumerate}
    \item Client reads file from disk
    \item Client encodes file content to Base64
    \item Client calls remote \texttt{upload\_file()} method via XML-RPC
    \item Server receives RPC call
    \item Server decodes Base64 data
    \item Server saves file to disk
    \item Server returns result message
    \item Client displays result
\end{enumerate}

\subsection{Technical Specifications}
\begin{itemize}
    \item \textbf{RPC Protocol}: XML-RPC
    \item \textbf{Transport}: HTTP
    \item \textbf{Port}: 5000
    \item \textbf{Host}: 0.0.0.0 (Server), 127.0.0.1 (Client)
    \item \textbf{Encoding}: Base64 (for binary data)
    \item \textbf{Data Format}: XML
\end{itemize}

\section{System Organization}

\subsection{Directory Structure}
\begin{verbatim}
Lab2/
├── server.py                  # RPC server
├── client.py                  # RPC client
├── test.txt                   # Test file
└── 02.rpc.file.transfer.tex   # LaTeX report
\end{verbatim}

\subsection{System Architecture}
\begin{itemize}
    \item \textbf{Server}: Exposes RPC methods, receives and saves files
    \item \textbf{Client}: Calls remote methods, sends files via RPC
    \item \textbf{XML-RPC Layer}: Handles serialization, HTTP transport, and method invocation
\end{itemize}

\subsection{Comparison: TCP vs RPC}
\begin{center}
\begin{tabular}{|l|l|l|}
\hline
\textbf{Aspect} & \textbf{TCP (Lab 1)} & \textbf{RPC (Lab 2)} \\
\hline
Abstraction & Low-level sockets & High-level function calls \\
Protocol & Custom binary & XML-RPC over HTTP \\
Complexity & Manual protocol design & Automatic serialization \\
Code Length & ~30 lines & ~20 lines \\
Error Handling & Manual & Built-in \\
Extensibility & Difficult & Easy to add methods \\
\hline
\end{tabular}
\end{center}

\section{Implementation}

\subsection{Server Code}
File \texttt{server.py} implements the RPC server:

\lstinputlisting[caption=server.py]{server.py}

\subsubsection{Server Code Explanation}
\begin{itemize}
    \item \textbf{Lines 1-2}: Import XML-RPC server and base64 modules
    \item \textbf{Lines 4-5}: Configure HOST and PORT
    \item \textbf{Lines 7-14}: Define \texttt{upload\_file()} RPC method
        \begin{itemize}
            \item Decode Base64 data to binary
            \item Write to file
            \item Return success/error message
        \end{itemize}
    \item \textbf{Lines 16-19}: Create RPC server, register method, and start serving
\end{itemize}

\subsection{Client Code}
File \texttt{client.py} implements the RPC client:

\lstinputlisting[caption=client.py]{client.py}

\subsubsection{Client Code Explanation}
\begin{itemize}
    \item \textbf{Lines 1-3}: Import XML-RPC client, base64, and os modules
    \item \textbf{Lines 5-6}: Configure HOST and PORT
    \item \textbf{Lines 8-12}: Get filename and check existence
    \item \textbf{Lines 14-15}: Read file and encode to Base64
    \item \textbf{Lines 17-19}: Create RPC proxy, call remote method, display result
\end{itemize}

\subsection{Key Differences from Lab 1}
\begin{itemize}
    \item \textbf{No manual socket management}: XML-RPC handles connections
    \item \textbf{No custom protocol}: Standard XML-RPC protocol
    \item \textbf{Base64 encoding}: Required for binary data in XML
    \item \textbf{Single RPC call}: Entire file sent in one method call
    \item \textbf{Simpler code}: Less boilerplate, more readable
\end{itemize}

\section{User Guide}

\subsection{System Requirements}
\begin{itemize}
    \item Python 3.x
    \item xmlrpc.server module (built-in)
    \item xmlrpc.client module (built-in)
    \item base64 module (built-in)
\end{itemize}

\subsection{How to Run}

\textbf{Step 1}: Open first terminal, run RPC Server:
\begin{verbatim}
cd ds2026/Lab2
python server.py
\end{verbatim}

\textbf{Step 2}: Open second terminal, run RPC Client:
\begin{verbatim}
cd ds2026/Lab2
python client.py
\end{verbatim}

\textbf{Step 3}: Enter the filename to send (e.g., \texttt{test.txt})

\subsection{Expected Results}
\begin{itemize}
    \item Server displays: "RPC Server listening on 0.0.0.0:5000"
    \item Client displays: "File test.txt uploaded successfully!"
    \item File is created in the Server directory
\end{itemize}

\section{Experimental Results}

\subsection{Test Case 1: Text File Transfer}
\begin{itemize}
    \item \textbf{Input}: test.txt (text file)
    \item \textbf{Result}: Success
    \item \textbf{Observation}: File transferred correctly via RPC
\end{itemize}

\subsection{Test Case 2: Binary File Transfer}
\begin{itemize}
    \item \textbf{Input}: image.png (binary file)
    \item \textbf{Result}: Success
    \item \textbf{Observation}: Base64 encoding preserves binary data integrity
\end{itemize}

\subsection{Test Case 3: Large File Transfer}
\begin{itemize}
    \item \textbf{Input}: large\_file.zip (10MB)
    \item \textbf{Result}: Success
    \item \textbf{Observation}: Single RPC call handles large files, but may have memory overhead
\end{itemize}

\subsection{Test Case 4: Non-existent File}
\begin{itemize}
    \item \textbf{Input}: nonexistent.txt
    \item \textbf{Result}: Client shows "File not found!"
    \item \textbf{Observation}: Error handling works correctly
\end{itemize}

\section{Who Does What}

\subsection{Client Responsibilities}
\begin{itemize}
    \item Read file from local filesystem
    \item Encode file content to Base64
    \item Create XML-RPC proxy connection
    \item Call remote \texttt{upload\_file()} method
    \item Display result to user
\end{itemize}

\subsection{Server Responsibilities}
\begin{itemize}
    \item Listen for RPC requests on specified port
    \item Expose \texttt{upload\_file()} method
    \item Decode Base64 data
    \item Save file to disk
    \item Return success/error message
\end{itemize}

\subsection{XML-RPC Framework Responsibilities}
\begin{itemize}
    \item Serialize method calls to XML
    \item Handle HTTP transport
    \item Deserialize XML responses
    \item Manage network connections
    \item Handle protocol-level errors
\end{itemize}

\section{Conclusion}

\subsection{Advantages of RPC Approach}
\begin{itemize}
    \item \textbf{Higher abstraction}: Focus on business logic, not network details
    \item \textbf{Simpler code}: ~20 lines vs ~30 lines in TCP version
    \item \textbf{Easier to extend}: Just add more RPC methods
    \item \textbf{Built-in error handling}: XML-RPC handles many edge cases
    \item \textbf{Standard protocol}: Interoperable with other languages
    \item \textbf{No custom protocol design}: Use proven XML-RPC standard
\end{itemize}

\subsection{Disadvantages of RPC Approach}
\begin{itemize}
    \item \textbf{Memory overhead}: Entire file loaded into memory and Base64 encoded
    \item \textbf{No streaming}: Cannot transfer very large files efficiently
    \item \textbf{XML overhead}: Larger payload size due to XML and Base64
    \item \textbf{Less control}: Cannot optimize low-level network behavior
    \item \textbf{Single point of failure}: One RPC call for entire transfer
\end{itemize}

\subsection{Lessons Learned}
\begin{itemize}
    \item Understanding RPC concepts and benefits
    \item Learning XML-RPC protocol in Python
    \item Comparing low-level (TCP) vs high-level (RPC) approaches
    \item Base64 encoding for binary data in text-based protocols
    \item Trade-offs between abstraction and control
\end{itemize}

\subsection{When to Use RPC vs TCP}
\begin{itemize}
    \item \textbf{Use RPC when}:
        \begin{itemize}
            \item Rapid development is priority
            \item File sizes are moderate (< 100MB)
            \item Need cross-language compatibility
            \item Want to expose multiple operations
        \end{itemize}
    \item \textbf{Use TCP when}:
        \begin{itemize}
            \item Transferring very large files
            \item Need maximum performance
            \item Require streaming capability
            \item Want full control over protocol
        \end{itemize}
\end{itemize}

\subsection{Future Enhancements}
\begin{itemize}
    \item Implement chunked RPC transfer for large files
    \item Add authentication and authorization
    \item Use JSON-RPC instead of XML-RPC for efficiency
    \item Add file compression before transfer
    \item Implement progress callback mechanism
    \item Add file listing and deletion RPC methods
    \item Use gRPC for better performance
\end{itemize}

\end{document}
